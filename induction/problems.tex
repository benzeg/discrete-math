\documentclass{article}
\usepackage{amsmath}
\usepackage{amssymb}
\usepackage{amsthm}

\begin{document}


Problem 1.

\begin{proof}
We prove by induction that for all integers $n > 10$, 
\[
n - 2 < \frac{n^2 - n}{12}.
\]

\textbf{Base Case: } $n = 11$
\[
11 - 2 < \frac{11^2 - 11}{12} \implies 9 < \frac{110}{12} \implies 9 < 9.166\quad \text{(True)}.
\]

\textbf{Inductive Step: } Assume for some $k > 10$ that
\[
k - 2 < \frac{k^2 - k}{12}.
\]
We show for $k + 1$:
\[
(k + 1) - 2 < \frac{(k + 1)^2 - (k + 1)}{12} \implies k - 1 < \frac{k^2 + k}{12}.
\]

\textbf{Proof: } From the inductive hypothesis,
\[
k - 2 < \frac{k^2 - k}{12}.
\]
Add 1 to both sides:
\[
k - 1 < \frac{k^2 - k}{12} + 1.
\]
Now observe that
\[
\frac{k^2 - k}{12} + 1 \leq \frac{k^2 + k}{12} \quad \text{if and only if} \quad 1 \leq \frac{2k}{12} \implies 6 \leq k.
\]
Since $k > 10$, this inequality holds. Therefore,
\[
k - 1 < \frac{k^2 - k}{12} + 1 \leq \frac{k^2 + k}{12},
\]
which completes the inductive step.

By mathematical induction, the statement holds for all integers $n > 10$.

\end{proof}

Problem 2.

\begin{proof}
We prove by induction that for all integers $n \geq 1$,
\[
\sum_{i=1}^{n} \sqrt{i} > \frac{2n\sqrt{n}}{3}.
\]

\textbf{Base Case: } $n = 1$
\[
\sum_{i=1}^{1} \sqrt{i} = \sqrt{1} = 1, \quad \frac{2 \cdot 1 \cdot \sqrt{1}}{3} = \frac{2}{3}, \quad 1 > \frac{2}{3}.
\]
The base case holds.

\textbf{Inductive Step: } Assume for some $k \geq 1$ that
\[
\sum_{i=1}^{k} \sqrt{i} > \frac{2k\sqrt{k}}{3}.
\]
We show for $k + 1$:
\[
\sum_{i=1}^{k+1} \sqrt{i} > \frac{2(k+1)\sqrt{k+1}}{3}.
\]

Starting from the left side:
\[
\sum_{i=1}^{k+1} \sqrt{i} = \sum_{i=1}^{k} \sqrt{i} + \sqrt{k+1} > \frac{2k\sqrt{k}}{3} + \sqrt{k+1}.
\]

It suffices to show that:
\[
\frac{2k\sqrt{k}}{3} + \sqrt{k+1} \geq \frac{2(k+1)\sqrt{k+1}}{3}.
\]

Multiply both sides by 3:
\[
2k\sqrt{k} + 3\sqrt{k+1} \geq 2(k+1)\sqrt{k+1}.
\]

Rearrange terms:
\[
2k\sqrt{k} \geq 2(k+1)\sqrt{k+1} - 3\sqrt{k+1} = (2k - 1)\sqrt{k+1}.
\]

Square both sides (valid since all terms are positive for $k \geq 1$):
\[
(2k\sqrt{k})^2 \geq ((2k - 1)\sqrt{k+1})^2,
\]
\[
4k^2 \cdot k \geq (2k - 1)^2 \cdot (k + 1),
\]
\[
4k^3 \geq (4k^2 - 4k + 1)(k + 1).
\]

Expand the right side:
\[
4k^3 \geq 4k^3 + 4k^2 - 4k^2 - 4k + k + 1 = 4k^3 - 3k + 1.
\]

Subtract $4k^3$ from both sides:
\[
0 \geq -3k + 1 \quad \Longleftrightarrow \quad 3k \geq 1 \quad \Longleftrightarrow \quad k \geq \frac{1}{3}.
\]

Since $k \geq 1$, this inequality holds. Therefore,
\[
\sum_{i=1}^{k+1} \sqrt{i} > \frac{2(k+1)\sqrt{k+1}}{3},
\]
which completes the inductive step.

By mathematical induction, the statement holds for all integers $n \geq 1$.
\end{proof}

Problem 3.
\begin{proof}
We prove by induction that for all integers $n \geq 0$,
\[
\sum_{i=0}^{n} i^2 = \frac{n(n+1)(2n+1)}{6}.
\]

\noindent\textbf{Base Case ($n = 0$):}
\[
\sum_{i=0}^{0} i^2 = 0 \quad \text{and} \quad \frac{0(0+1)(2\cdot0+1)}{6} = 0.
\]
Thus, the base case holds.

\noindent\textbf{Inductive Step:} Assume that for some $k \geq 0$,
\[
\sum_{i=0}^{k} i^2 = \frac{k(k+1)(2k+1)}{6}.
\]
We will show that the statement holds for $k+1$, that is,
\[
\sum_{i=0}^{k+1} i^2 = \frac{(k+1)(k+2)(2k+3)}{6}.
\]

Starting from the left-hand side:
\begin{align*}
\sum_{i=0}^{k+1} i^2 &= \sum_{i=0}^{k} i^2 + (k+1)^2 \\
&= \frac{k(k+1)(2k+1)}{6} + (k+1)^2 \\
&= (k+1)\left[\frac{k(2k+1)}{6} + (k+1)\right] \\
&= (k+1)\left[\frac{k(2k+1) + 6(k+1)}{6}\right] \\
&= (k+1)\left[\frac{2k^2 + k + 6k + 6}{6}\right] \\
&= (k+1)\left[\frac{2k^2 + 7k + 6}{6}\right] \\
&= \frac{(k+1)(k+2)(2k+3)}{6}.
\end{align*}
This completes the inductive step.

\noindent By mathematical induction, the statement holds for all integers $n \geq 0$.
\end{proof}

Problem 4.
\begin{proof}
We prove by \textbf{strong induction} that every integer $n > 1$ is either prime or can be written as a product of prime numbers.

\textbf{Base Case:} ($n = 2$)
The number $2$ is prime by definition, as its only positive divisors are $1$ and $2$. A single prime is trivially considered a product of primes. Thus, the statement holds for $n=2$.

\textbf{Inductive Step:}
Assume the inductive hypothesis: that for some integer $k \geq 2$, every integer $j$ with $2 \leq j \leq k$ is a product of primes (i.e., is prime itself or can be factored into primes). We must show that the integer $k+1$ is also a product of primes.

We consider two cases:

\begin{itemize}
    \item \textbf{Case 1:} If $k+1$ is prime, then it is trivially a product of primes (itself), and we are done.
    \item \textbf{Case 2:} If $k+1$ is composite, then by the definition of a composite number, it has positive divisors other than $1$ and itself. Therefore, it can be written as:
    \[
    k+1 = a \cdot b
    \]
    where $a$ and $b$ are integers satisfying $1 < a, b < k+1$.
    
    Since $2 \leq a \leq k$ and $2 \leq b \leq k$, the strong induction hypothesis applies to both $a$ and $b$. Hence, both are products of primes:
    \begin{align*}
        a &= p_1p_2 \cdots p_m, \\
        b &= q_1q_2 \cdots q_n,
    \end{align*}
    where each $p_i$ and $q_j$ is a prime number.
    
    Substituting these products, we find:
    \[
    k+1 = a \cdot b = (p_1p_2 \cdots p_m)(q_1q_2 \cdots q_n).
    \]
    This is clearly a product of prime numbers.
\end{itemize}

In both cases, $k+1$ is a product of primes. By the principle of strong mathematical induction, every integer $n > 1$ is a product of primes.
\end{proof}

Problem 6.
\begin{proof}
We prove by induction that for all positive integers $n$,
\[
F_1 + F_3 + \cdots + F_{2n-1} = F_{2n},
\]
where $F_i$ denotes the $i^{\text{th}}$ Fibonacci number, defined by $F_1 = 1$, $F_2 = 1$, and $F_n = F_{n-1} + F_{n-2}$ for $n \geq 3$.

\textbf{Base Case:} ($n = 1$)
\[
F_1 = 1 \quad \text{and} \quad F_2 = 1, \quad \text{so} \quad F_1 = F_2.
\]
Thus, the base case holds.

\textbf{Inductive Hypothesis:} Assume for some $k \geq 1$ that
\[
F_1 + F_3 + \cdots + F_{2k-1} = F_{2k}.
\]

\textbf{Inductive Step:} We now show that the statement holds for $n = k+1$, i.e.,
\[
F_1 + F_3 + \cdots + F_{2k-1} + F_{2k+1} = F_{2k+2}.
\]

Starting from the left-hand side:
\begin{align*}
F_1 + F_3 + \cdots + F_{2k-1} + F_{2k+1} &= \left(F_1 + F_3 + \cdots + F_{2k-1}\right) + F_{2k+1} \\
&= F_{2k} + F_{2k+1} \quad \text{(by the inductive hypothesis)} \\
&= F_{2k+2} \quad \text{(by the Fibonacci recurrence relation)}.
\end{align*}
This completes the inductive step.

By mathematical induction, the statement holds for all positive integers $n$.
\end{proof}

Problem 7.
\begin{proof}
We will prove by induction that $2^{n} > n^{2}$ for all natural numbers $n \geq 5$.

\textbf{Base Case: } $n = 5$
\[
2^{5} = 32 \quad \text{and} \quad 5^{2} = 25.
\]
Since $32 > 25$, the base case holds.

\textbf{Inductive Step:} Assume the induction hypothesis holds for some integer $k \geq 5$, that is, assume
\[
2^{k} > k^{2}. \tag{IH}
\]
We must now prove that the inequality holds for $k+1$, i.e.,
\[
2^{k+1} > (k+1)^{2}.
\]

Starting with the left-hand side of the desired inequality:
\[
2^{k+1} = 2 \cdot 2^{k}.
\]
By the induction hypothesis (IH), $2^{k} > k^{2}$, so we can substitute:
\[
2^{k+1} > 2 \cdot k^{2}. \tag{1}
\]

Our goal is to show that the right-hand side of (1) is greater than $(k+1)^{2}$. Let us therefore examine the inequality:
\[
2k^{2} > (k+1)^{2}.
\]
Expanding the right-hand side gives:
\[
2k^{2} > k^{2} + 2k + 1.
\]
Subtracting $k^2$ from both sides yields the equivalent inequality:
\[
k^{2} > 2k + 1. \tag{2}
\]

We now show that inequality (2) is true for $k \geq 5$. Consider the function $f(k) = k^{2} - 2k - 1$. Its derivative $f'(k) = 2k - 2$ is positive for $k > 1$, so $f(k)$ is increasing for $k \geq 5$. Since $f(5) = 25 - 10 - 1 = 14 > 0$, it follows that $k^{2} > 2k + 1$ for all $k \geq 5$. Therefore, inequality (2) holds.

We can now chain the inequalities together. From (1) we have $2^{k+1} > 2k^{2}$, and from (2) we have $2k^{2} > k^{2} + 2k + 1 = (k+1)^{2}$. Hence,
\[
2^{k+1} > 2k^{2} > (k+1)^{2},
\]
which completes the inductive step.

By the principle of mathematical induction, $2^{n} > n^{2}$ for all natural numbers $n \geq 5$.
\end{proof}

Problem 9.

\begin{proof}
We prove by induction that for all integers $n \geq 1$, $9^n - 2^n$ is divisible by $7$.

\textbf{Base Case ($n = 1$):}
\[
9^1 - 2^1 = 9 - 2 = 7
\]
Since $7$ is divisible by $7$, the base case holds.

\textbf{Inductive Step:} Assume that for some integer $k \geq 1$, the statement holds. That is, assume
\[
9^k - 2^k = 7i \quad \text{for some integer } i.
\]
This is our inductive hypothesis. We will show that the statement is true for $k+1$, i.e., that $9^{k+1} - 2^{k+1}$ is divisible by $7$.

We begin with the expression for $k+1$ and manipulate it to incorporate the inductive hypothesis:
\begin{align*}
9^{k+1} - 2^{k+1} &= 9 \cdot 9^k - 2 \cdot 2^k \\
&= 9 \cdot 9^k - 9 \cdot 2^k + 9 \cdot 2^k - 2 \cdot 2^k && \text{(Adding and subtracting } 9 \cdot 2^k \text{)} \\
&= 9(9^k - 2^k) + 2^k(9 - 2) \\
&= 9(7i) + 2^k \cdot 7 && \text{(By the inductive hypothesis)} \\
&= 7(9i + 2^k)
\end{align*}
Let $j = 9i + 2^k$, which is an integer. Therefore, we have shown that
\[
9^{k+1} - 2^{k+1} = 7j,
\]
which is divisible by $7$.

By the principle of mathematical induction, $9^n - 2^n$ is divisible by $7$ for all integers $n \geq 1$.
\end{proof}    

Problem 10.
\begin{proof}
We will prove by induction that \(\sum_{i=1}^{n} (4i - 3) = \frac{n(4n - 2)}{2}\) for all natural numbers \(n\).

\textbf{Base Case (\(n = 1\)):}
\[
\sum_{i=1}^{1} (4i - 3) = 4(1) - 3 = 1.
\]
\[
\frac{1 \cdot (4 \cdot 1 - 2)}{2} = \frac{1 \cdot 2}{2} = 1.
\]
Since both expressions equal \(1\), the base case holds.

\textbf{Inductive Step:} Assume that for some integer \(k \geq 1\), the statement holds. That is, assume the inductive hypothesis:
\[
\sum_{i=1}^{k} (4i - 3) = \frac{k(4k - 2)}{2}.
\]
We will show that the statement is true for \(k+1\), i.e.,
\[
\sum_{i=1}^{k+1} (4i - 3) = \frac{(k+1)(4(k+1) - 2)}{2} = \frac{(k+1)(4k + 2)}{2}.
\]

Starting from the left-hand side of the statement for \(k+1\):
\begin{align*}
\sum_{i=1}^{k+1} (4i - 3) &= \left( \sum_{i=1}^{k} (4i - 3) \right) + (4(k+1) - 3) \\
&= \frac{k(4k - 2)}{2} + (4k + 4 - 3) && \text{(by the inductive hypothesis)} \\
&= \frac{k(4k - 2)}{2} + (4k + 1) \\
&= \frac{k(4k - 2) + 2(4k + 1)}{2} && \text{(obtaining a common denominator)} \\
&= \frac{4k^2 - 2k + 8k + 2}{2} \\
&= \frac{4k^2 + 6k + 2}{2}.
\end{align*}
Now, we factor the numerator to show it matches our target expression:
\begin{align*}
\frac{4k^2 + 6k + 2}{2} &= \frac{2(2k^2 + 3k + 1)}{2} \\
&= 2k^2 + 3k + 1.
\end{align*}
And indeed, the target expression also simplifies to the same result:
\begin{align*}
\frac{(k+1)(4k + 2)}{2} &= \frac{(k+1) \cdot 2(2k + 1)}{2} \\
&= (k+1)(2k + 1) \\
&= 2k^2 + k + 2k + 1 \\
&= 2k^2 + 3k + 1.
\end{align*}
Therefore, we have shown that
\[
\sum_{i=1}^{k+1} (4i - 3) = \frac{(k+1)(4k + 2)}{2},
\]
which completes the inductive step.

By the principle of mathematical induction, the formula \(\sum_{i=1}^{n} (4i - 3) = \frac{n(4n - 2)}{2}\) holds for all natural numbers \(n\).
\end{proof}
Problem 11.
\begin{proof}
We prove by induction that $2^{n} \geq 1 + n$ for all $n \geq 1$.

\textbf{Base Case ($n = 1$):}
\[
2^{1} = 2 \geq 1 + 1 = 2.
\]
Thus, the base case holds.

\textbf{Inductive Step:} Assume that for some integer $k \geq 1$, the statement holds. That is, assume
\[
2^{k} \geq 1 + k.
\]
This is our inductive hypothesis. We will show that the statement is true for $k+1$, i.e., that $2^{k+1} \geq 1 + (k+1)$.

\begin{align*}
2^{k+1} &= 2^{k} \cdot 2 \\
&\geq (1 + k) \cdot 2 \quad \text{(by the inductive hypothesis)} \\
&= 2 + 2k \\
&= (k + 2) + k \\
&\geq k + 2 \quad \text{(since $k \geq 1 > 0$)} \\
&= 1 + (k + 1).
\end{align*}
Thus, the statement holds for $k+1$.

By mathematical induction, $2^{n} \geq 1 + n$ for all $n \geq 1$.
\end{proof}

Problem 12.
\begin{proof}
We prove by induction that $3^{n} < (n + 1)!$ for all $n \geq 4$.

\textbf{Base Case ($n = 4$):}
\[
3^{4} = 81 \quad \text{and} \quad (4 + 1)! = 5! = 5 \times 4 \times 3 \times 2 \times 1 = 120.
\]
Since $81 < 120$, the base case holds.

\textbf{Inductive Step:} Assume that for some integer $k \geq 4$, the statement holds. That is, assume
\[
3^{k} < (k + 1)!.
\]
This is our inductive hypothesis. We will show that the statement is true for $k + 1$, i.e., that 
\[
3^{k+1} < ((k + 1) + 1)! = (k + 2)!.
\]

\begin{align*}
3^{k+1} &= 3^{k} \cdot 3 \\
&< (k + 1)! \cdot 3 \quad \text{(by the inductive hypothesis)} \\
&< (k + 1)! \cdot (k + 2) \quad \text{(since $k \geq 4$, we have $3 < k + 2$)} \\
&= (k + 2)!.
\end{align*}
Thus, the statement holds for $k + 1$.

By mathematical induction, $3^{n} < (n + 1)!$ for all $n \geq 4$.
\end{proof}
\end{document}
