\documentclass{article}
\usepackage{amsmath}
\usepackage{amssymb}
\usepackage{amsthm}

\begin{document}


Problem 1.

\begin{proof}
We prove by induction that for all integers $n > 10$, 
\[
n - 2 < \frac{n^2 - n}{12}.
\]

\textbf{Base Case: } $n = 11$
\[
11 - 2 < \frac{11^2 - 11}{12} \implies 9 < \frac{110}{12} \implies 9 < 9.166\quad \text{(True)}.
\]

\textbf{Inductive Step: } Assume for some $k > 10$ that
\[
k - 2 < \frac{k^2 - k}{12}.
\]
We show for $k + 1$:
\[
(k + 1) - 2 < \frac{(k + 1)^2 - (k + 1)}{12} \implies k - 1 < \frac{k^2 + k}{12}.
\]

\textbf{Proof: } From the inductive hypothesis,
\[
k - 2 < \frac{k^2 - k}{12}.
\]
Add 1 to both sides:
\[
k - 1 < \frac{k^2 - k}{12} + 1.
\]
Now observe that
\[
\frac{k^2 - k}{12} + 1 \leq \frac{k^2 + k}{12} \quad \text{if and only if} \quad 1 \leq \frac{2k}{12} \implies 6 \leq k.
\]
Since $k > 10$, this inequality holds. Therefore,
\[
k - 1 < \frac{k^2 - k}{12} + 1 \leq \frac{k^2 + k}{12},
\]
which completes the inductive step.

By mathematical induction, the statement holds for all integers $n > 10$.

\end{proof}

Problem 2.

\begin{proof}
We prove by induction that for all integers $n \geq 1$,
\[
\sum_{i=1}^{n} \sqrt{i} > \frac{2n\sqrt{n}}{3}.
\]

\textbf{Base Case: } $n = 1$
\[
\sum_{i=1}^{1} \sqrt{i} = \sqrt{1} = 1, \quad \frac{2 \cdot 1 \cdot \sqrt{1}}{3} = \frac{2}{3}, \quad 1 > \frac{2}{3}.
\]
The base case holds.

\textbf{Inductive Step: } Assume for some $k \geq 1$ that
\[
\sum_{i=1}^{k} \sqrt{i} > \frac{2k\sqrt{k}}{3}.
\]
We show for $k + 1$:
\[
\sum_{i=1}^{k+1} \sqrt{i} > \frac{2(k+1)\sqrt{k+1}}{3}.
\]

Starting from the left side:
\[
\sum_{i=1}^{k+1} \sqrt{i} = \sum_{i=1}^{k} \sqrt{i} + \sqrt{k+1} > \frac{2k\sqrt{k}}{3} + \sqrt{k+1}.
\]

It suffices to show that:
\[
\frac{2k\sqrt{k}}{3} + \sqrt{k+1} \geq \frac{2(k+1)\sqrt{k+1}}{3}.
\]

Multiply both sides by 3:
\[
2k\sqrt{k} + 3\sqrt{k+1} \geq 2(k+1)\sqrt{k+1}.
\]

Rearrange terms:
\[
2k\sqrt{k} \geq 2(k+1)\sqrt{k+1} - 3\sqrt{k+1} = (2k - 1)\sqrt{k+1}.
\]

Square both sides (valid since all terms are positive for $k \geq 1$):
\[
(2k\sqrt{k})^2 \geq ((2k - 1)\sqrt{k+1})^2,
\]
\[
4k^2 \cdot k \geq (2k - 1)^2 \cdot (k + 1),
\]
\[
4k^3 \geq (4k^2 - 4k + 1)(k + 1).
\]

Expand the right side:
\[
4k^3 \geq 4k^3 + 4k^2 - 4k^2 - 4k + k + 1 = 4k^3 - 3k + 1.
\]

Subtract $4k^3$ from both sides:
\[
0 \geq -3k + 1 \quad \Longleftrightarrow \quad 3k \geq 1 \quad \Longleftrightarrow \quad k \geq \frac{1}{3}.
\]

Since $k \geq 1$, this inequality holds. Therefore,
\[
\sum_{i=1}^{k+1} \sqrt{i} > \frac{2(k+1)\sqrt{k+1}}{3},
\]
which completes the inductive step.

By mathematical induction, the statement holds for all integers $n \geq 1$.
\end{proof}

Problem 3.
\begin{proof}
We prove by induction that for all integers $n \geq 0$,
\[
\sum_{i=0}^{n} i^2 = \frac{n(n+1)(2n+1)}{6}.
\]

\noindent\textbf{Base Case ($n = 0$):}
\[
\sum_{i=0}^{0} i^2 = 0 \quad \text{and} \quad \frac{0(0+1)(2\cdot0+1)}{6} = 0.
\]
Thus, the base case holds.

\noindent\textbf{Inductive Step:} Assume that for some $k \geq 0$,
\[
\sum_{i=0}^{k} i^2 = \frac{k(k+1)(2k+1)}{6}.
\]
We will show that the statement holds for $k+1$, that is,
\[
\sum_{i=0}^{k+1} i^2 = \frac{(k+1)(k+2)(2k+3)}{6}.
\]

Starting from the left-hand side:
\begin{align*}
\sum_{i=0}^{k+1} i^2 &= \sum_{i=0}^{k} i^2 + (k+1)^2 \\
&= \frac{k(k+1)(2k+1)}{6} + (k+1)^2 \\
&= (k+1)\left[\frac{k(2k+1)}{6} + (k+1)\right] \\
&= (k+1)\left[\frac{k(2k+1) + 6(k+1)}{6}\right] \\
&= (k+1)\left[\frac{2k^2 + k + 6k + 6}{6}\right] \\
&= (k+1)\left[\frac{2k^2 + 7k + 6}{6}\right] \\
&= \frac{(k+1)(k+2)(2k+3)}{6}.
\end{align*}
This completes the inductive step.

\noindent By mathematical induction, the statement holds for all integers $n \geq 0$.
\end{proof}

Problem 4.
\begin{proof}
We prove by \textbf{strong induction} that every integer $n > 1$ is either prime or can be written as a product of prime numbers.

\textbf{Base Case:} ($n = 2$)
The number $2$ is prime by definition, as its only positive divisors are $1$ and $2$. A single prime is trivially considered a product of primes. Thus, the statement holds for $n=2$.

\textbf{Inductive Step:}
Assume the inductive hypothesis: that for some integer $k \geq 2$, every integer $j$ with $2 \leq j \leq k$ is a product of primes (i.e., is prime itself or can be factored into primes). We must show that the integer $k+1$ is also a product of primes.

We consider two cases:

\begin{itemize}
    \item \textbf{Case 1:} If $k+1$ is prime, then it is trivially a product of primes (itself), and we are done.
    \item \textbf{Case 2:} If $k+1$ is composite, then by the definition of a composite number, it has positive divisors other than $1$ and itself. Therefore, it can be written as:
    \[
    k+1 = a \cdot b
    \]
    where $a$ and $b$ are integers satisfying $1 < a, b < k+1$.
    
    Since $2 \leq a \leq k$ and $2 \leq b \leq k$, the strong induction hypothesis applies to both $a$ and $b$. Hence, both are products of primes:
    \begin{align*}
        a &= p_1p_2 \cdots p_m, \\
        b &= q_1q_2 \cdots q_n,
    \end{align*}
    where each $p_i$ and $q_j$ is a prime number.
    
    Substituting these products, we find:
    \[
    k+1 = a \cdot b = (p_1p_2 \cdots p_m)(q_1q_2 \cdots q_n).
    \]
    This is clearly a product of prime numbers.
\end{itemize}

In both cases, $k+1$ is a product of primes. By the principle of strong mathematical induction, every integer $n > 1$ is a product of primes.
\end{proof}
\end{document}
